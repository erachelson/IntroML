\documentclass{beamer}

\mode<presentation>
{
\usetheme{Copenhagen}
%\usetheme{Boadilla}
%\usecolortheme{seahorse}
%\useoutertheme{infolines}
%\useoutertheme[compress]{miniframes}
%\setbeamercovered{transparent}
}

%% \mode<presentation>
%% {
%% \usetheme{progressbar}
%% \setbeamercovered{transparent}
%% }

%\usepackage[english]{babel}
\usepackage[english]{babel}
\usepackage[utf8]{inputenc}
\usepackage[T1]{fontenc}

\usepackage{mathptmx}
\usepackage[scaled=.90]{helvet}
\usepackage[T1]{fontenc}
\usepackage{xspace}
%\usepackage{appendixnumberbeamer}
\usepackage[noend]{algorithmic}
%\usepackage[algo2e,vlined,algochapter,ruled,dotocloa]{algorithm2e}
\usepackage{fancybox}
\usepackage{algorithm}
\usepackage[noend]{algorithmic}
\usepackage{amssymb,amsmath}
% \usepackage{ulem}
\usepackage{makecell}
\usepackage{times}
\usepackage{bbding}

\title[A few topics in Reinforcement Learning]{From Field problems to Machine Learning}
\subtitle{An introduction to the Data Science workflow\\and a motivation to understand Machine Learning}
%\title{The Optimal Swapping Problem during Nuclear Refueling Operations}


\author{E. Rachelson}

\institute{\includegraphics[width=1.5cm]{img/isae.jpg}}

\date{}


% This is only inserted into the PDF information catalog. Can be left
% out.
\subject{From Predictive Maintenance to Machine Learning}

\setbeamerfont{bibliography entry author}{shape=\upshape,series=\bfseries,size=\footnotesize}%
\setbeamerfont{bibliography entry title}{shape=\upshape,size=\scriptsize,series=\mdseries}
\setbeamerfont{bibliography entry journal}{shape=\upshape,size=\scriptsize,series=\mdseries}
\setbeamerfont{bibliography entry note}{shape=\upshape,size=\scriptsize,series=\mdseries}

\setbeamercolor{block}{bg=blue,fg=red}

\beamertemplatenavigationsymbolsempty
\setbeamertemplate{footline}[frame number]

\begin{document}

\begin{frame}[plain]
\titlepage
\end{frame}

\begin{frame}{Schedule}
\begin{enumerate}
\item General introduction and motivation\\
{\small \it How does ML fit within your business process.\\
Why you should take time to understand what's under the hood in ML.}
\item The importance of data pre-processing\\
{\small \it A practical illustration.}
\item A geometrical approach to ML\\
{\small \it Support Vector Machines and a bit of kernel theory.}
\item A probabilistic approach to ML\\
{\small \it Naive Bayes Classification and Gaussian Processes.}
\item A connectionist perspective \\
{\small \it From mimicking the human brain to Artificial Neural Networks.\\
Deep Learning.}
\item Ensembles of explainable models\\
{\small \it Decision Trees, Bagging and Random Forests.}
\item Practical session, discussion and conclusion
\end{enumerate}
\end{frame}

\begin{frame}{Course goals}
\small
By the end of the class, you should be able to:
\begin{itemize}
\item implement a generic workflow of data analysis for your application field;
\item know the main bottlenecks and challenges of data-driven approaches;
\item link some field problems to their formal Machine Learning counterparts;
\item know the main categories of Machine Learning algorithms and which formal problem they solve;
\item know the name and principles of some key methods in Machine Learning:
\begin{itemize}
\item SVM,
\item kernel methods,
\item Naive Bayes Classification,
\item Gaussian Processes,
\item Artificial Neural Networks,
\item Deep Learning,
\item Random Forests;
\end{itemize} 
\item know the existence of scikit-learn and its API.
\end{itemize}
\end{frame}

\begin{frame}{Course material}
\small\url{https://github.com/erachelson/IntroML}
\end{frame}

\begin{frame}{Case studies}
Let's list and discuss some cases from your experience and from the literature.
\begin{itemize}
\item ~[Your cases here!]\pause
\item Predictive maintenance%(RUL/TTF prediction, proba of failure, anomaly detection)
\item Market segmentation%(grouping customer habits)
\item Demand forecast
\item Preliminary design studies
\item Clinical diagnosis
\item Documentation management
\item Satellite imaging
\end{itemize}
\end{frame}

\begin{frame}{From tasks, to data, to ML}
For all these cases, let's fill the table below, to build a common understanding of:
\begin{itemize}
\item the nature of data at stake
\item the different tasks to automate
\item the difficulties
\end{itemize}
~\\
~\\
\footnotesize
\begin{tabular}{|c|c|c|c|c|c|}
\hline
\thead{\begin{minipage}{1.3cm}\centering Use case\end{minipage}} & \thead{\begin{minipage}{1.3cm}\centering Type of data\end{minipage}} & \begin{minipage}{1.3cm}\centering Properties of data\end{minipage} & \begin{minipage}{1.3cm}\centering Task to automate\end{minipage} & \begin{minipage}{1.3cm}\centering Difficulties\end{minipage} & \begin{minipage}{1.3cm}\centering Comments\end{minipage}\\
\hline
 & & & & & \\
\hline
 & & & & & \\
\hline
 & & & & & \\
\hline
 & & & & & \\
\hline
\end{tabular}
\end{frame}

\begin{frame}{The Data Scientist perspective}
\centering \includegraphics[width=6cm]{img/How-can-i-help.jpg}
\end{frame}

\begin{frame}{Identified needs}
Let's take the example of Predictive Maintenance.\\
~\\
We would like to build automated tools for the following tasks:
\begin{itemize}
\item Visualize system state
\item Identify anomalies
\item Predict Remaining Useful Life (RUL) / Time To Failure (TTF)
\item Predict failure occurrence or probability at a given horizon
\end{itemize}
All this, in order to base our maintenance strategy on the (inferred) system state, rather than a general statistical trend.\\
~\\
\begin{block}{}
Can you relate this task decomposition to the other use-cases we've seen earlier?
\end{block}
Traditionaly, all this is based on user expertise.\\
Let's take a data-driven approach.
\end{frame}

\begin{frame}{Data analysis workflow}
\begin{enumerate}
\item<1-> Collect
\item<2-> Analyze
\item<3-> Predict
\item<4-> Decide
\end{enumerate}
\begin{overlayarea}{10cm}{4cm}
\begin{block}{}
\only<1>{
\begin{itemize}
\item Sensors deployment
\item Historical data collection
\item Integrated storage (datawarehouses) and retrieval issues
\end{itemize}
$\rightarrow$ Extract-Transform-Load (ETL) process\\
More on ETL: \href{https://www.sas.com/en_us/insights/data-management/what-is-etl.html}{[link]}.\\
The \emph{data engineer}'s job: data quality, management, availability.
}
\only<2>{
\begin{itemize}
\item data cleaning
\item feature selection / engineering
\item performance criteria
\item algorithm selection
\item parameters tuning
\end{itemize}
The \emph{data analyst} or \emph{data scientist}'s job.\\
But can't be disconnected from field engineers on the task.
}
\only<3>{
\begin{itemize}
\item Make predictions on new test cases
\item Deploy solution in your operational process
\item Make things usable
\end{itemize}
}
\only<4>{
\begin{itemize}
\item Improve your decisions
\end{itemize}
End-user.\\
Job title depends on your professional field.
}
\only<5>{
Need to automate as many steps as possible in this workflow\\
\hspace{2cm}$\rightarrow$ data-driven approaches\\
\hspace{2cm}$\rightarrow$ Machine Learning for step 2 (and 3)
}
\end{block}
\end{overlayarea}
\end{frame}

\begin{frame}{A word on data quality}
\begin{itemize}
\item amount of data: data is often abundant but crucial data is often scarce
\item noise, errors, missing data, outdated data: reliability
\item high-dimensional data
\item class imbalance
\item heterogeneous data (scalars, booleans, time series, images, text, \ldots)
\end{itemize}
All these will influence your algorithmic design or choices.\\
~\\
So let's talk about algorithms to see how we can solve the problems listed earlier.
\end{frame}

\begin{frame}{Machine Learning}
\centering
Machines that learn?\\
Let's try to give a general definition.\\
~\\
\visible<2>{Machine learning is a field of computer science that gives computer systems the ability to ``learn'' (i.e. progressively improve performance on a specific task) with data, without being explicitly programmed.
\begin{flushright}
{\footnotesize (\href{https://en.wikipedia.org/wiki/Machine_learning}{Wikipedia)}}
\end{flushright}}
\end{frame}

\begin{frame}{ML examples}
\begin{overlayarea}{\textwidth}{5cm}
\begin{itemize}
\item<1-> Given 20 years of clinical data, will this patient have a second heart attack in the next 5 years?
\item<2-> What price for this stock, 6 months from now?
\item<3-> Is this handwritten number a 7?
\item<4-> Is this e-mail a spam?
\item<5-> Can I cluster together customers? press articles? genes?
\item<6-> What is the best strategy when playing Counter Strike? or poker?
\end{itemize}
\end{overlayarea}
\begin{overlayarea}{\textwidth}{2.5cm}
\begin{center}
\only<1>{\includegraphics[height=2.5cm]{img/surgery.jpg}}
\only<2>{\includegraphics[height=2.5cm]{img/finance.png}}
\only<3>{\includegraphics[height=2.5cm]{img/seven.jpg}}
\only<4>{\begin{tabular}{rl}
\begin{minipage}{0.9cm}
\includegraphics[height=1cm]{img/message.jpg}
\end{minipage} & \textbf{Enlarge your thesis!}
\end{tabular}}
\only<5>{\includegraphics[height=2.5cm]{img/people.jpg}\includegraphics[height=2.5cm]{img/words.jpg}\includegraphics[height=2.5cm]{img/DNA_microarray.pdf}}
\only<6>{\includegraphics[height=2.5cm]{img/counterstrike2.jpg}~~\includegraphics[height=2.5cm]{img/poker2.jpg}}
\end{center}
\end{overlayarea}
\begin{overlayarea}{\textwidth}{1cm}
\tiny Image sources: %
\only<1>{\href{https://commons.wikimedia.org/wiki/File:Cardiac_surgery_operating_room.jpg}{Wikimedia commons}}%
\only<2>{\href{https://commons.wikimedia.org/wiki/File:Crypto_Composite_home_page.png}{Wikimedia commons}}%
\only<3>{\href{http://www.urbanthreads.com/productImages/regularSize/UTH4670.jpg}{[link]}}%
\only<4>{\href{https://www.iconfinder.com/icons/118781/mail_message_new_icon}{Iconfinder}}%
\only<5>{\href{http://clipart-library.com/white-people-cliparts.html}{People.jpg} / 
\href{http://journals.sagepub.com/doi/pdf/10.1207/s15328023top1701_10}{Writing to Discuss: Use of a Clustering Technique} / 
\href{https://commons.wikimedia.org/wiki/File:DNA_microarray.svg}{DNA microarray}}
\only<6>{\href{https://en.wikipedia.org/wiki/File:Counter-Strike_Source_(box_art).jpg}{CS:source} / \href{https://commons.wikimedia.org/wiki/File:2006_WSOP_Main_Event_Table.jpg}{poker}}
\end{overlayarea}
\end{frame}

\begin{frame}{ML tasks}
What does ML do? 3 main tasks.\\
~\\
\begin{tabular}{|c|c|c|c|}
\hline
\makecell[{{p{0.07\textwidth}}}]{\textbf{Task}} &
\makecell[{{p{0.27\textwidth}}}]{\centering Supervised\\ Learning} & \makecell[{{p{0.27\textwidth}}}]{\centering Unsupervised Learning} & \makecell[{{p{0.27\textwidth}}}]{\centering Reinforcement Learning}\\
\hline
\makecell[{{p{0.07\textwidth}}}]{\textbf{Goal}} &
\makecell[{{p{0.27\textwidth}}}]{\centering Learn a function, $f(x)=y$} & \makecell[{{p{0.27\textwidth}}}]{\centering Find groups and correlations, $x\in C$} & \makecell[{{p{0.27\textwidth}}}]{\centering Optimal control, $f(x)=u \ / \ \max\sum r$}\\
\hline
\makecell[{{p{0.07\textwidth}}}]{\textbf{Data}} &
\makecell[{{p{0.27\textwidth}}}]{\centering $\{(x,y)\}$} & \makecell[{{p{0.27\textwidth}}}]{\centering $\{x\}$} & \makecell[{{p{0.27\textwidth}}}]{\centering $\{(x,u,r,x')\}$}\\
\hline
\makecell[{{p{0.07\textwidth}}}]{\textbf{Sub-task}} &
\makecell[{{p{0.27\textwidth}}}]{\centering Classification, Regression} & \makecell[{{p{0.27\textwidth}}}]{\centering Clustering, Density estimation, Dimensionnality reduction} & \makecell[{{p{0.27\textwidth}}}]{\centering Value estimation, Policy optimization}\\
\hline
\makecell[{{p{0.07\textwidth}}}]{\textbf{Algo ex.}} &
\makecell[{{p{0.27\textwidth}}}]{\centering Neural Networks, SVM, Random Forests} & \makecell[{{p{0.27\textwidth}}}]{\centering k-means, PCA, HCA} & \makecell[{{p{0.27\textwidth}}}]{\centering Q-learning}\\
\hline
\end{tabular}
\end{frame}

\begin{frame}{Evaluation criteria}
Evaluating ML methods? What do we really want?\\
~\\
\begin{tabular}{ll}
\begin{minipage}{0.65\textwidth}
Ability to fit the training data:
\begin{itemize}
\item Regression: Mean Square Error
\item Classification: Accuracy, TP, FP, ROC, AUC\ldots\\
cf. {\href{https://en.wikipedia.org/wiki/Precision_and_recall}{this Wikipedia article}}
\item Clustering: similarity scores
\end{itemize}
~\\
Ability to generalize:
\begin{itemize}
\item Goal: filter out noise, avoid overfitting, generalize to unseen cases.
\item ML Notions:
\begin{itemize}
\item maximize margin
\item minimize difference btw class distributions (cross-entropy)
\end{itemize}
\end{itemize}
\end{minipage} &
\begin{minipage}{0.3\textwidth}
\includegraphics[width=\textwidth]{img/Precisionrecall.png}\\
\tiny Image source: \href{https://commons.wikimedia.org/wiki/File:Precisionrecall.svg}{Wikimedia commons}
\end{minipage}
\end{tabular}
\end{frame}

\begin{frame}{Misconceptions and clarifications}
\begin{itemize}
\item[AI] ML is only a small (currently fashionable) part of Artificial Intelligence.
\item[BD] Big Data refers to working with datasets that have large Volume, Variety, Velocity (, Veracity, and Value).
\item[DL] Deep Learning is Machine Learning with Deep Neural Networks.
\item[threat] [Provocative thought] ML / Data Science / Big Data are as much of a threat (to jobs, the society, the economy\ldots) as the combustion engine was in the XIXth century.
\item[ethics] Technical problems are not just technical problems and solutions \emph{always} imply some tradeoff. Who bears the (moral) responsibility?
\end{itemize}
\begin{center}
	\includegraphics[width=5cm]{img/datascientist.png}
\end{center}
\end{frame}

\begin{frame}{ML software}
Software:
\begin{itemize}
\item Many free software libraries: scikit-learn, tensorflow, pytorch\ldots check \url{www.mloss.org} if you're curious.
\item Free environments: Weka, RStudio\ldots
\item Commercial embedded solutions (more or less specialized): Matlab, IBM, Microsoft\ldots
\end{itemize}
\vspace{2em}
Short ``time to market'' and high innovation pace.
\end{frame}

\begin{frame}{The process of (Un)Supervised Learning}
\begin{center}
\includegraphics[height=7cm]{img/process.png}
\end{center}
\vspace{-0.5cm}
{\footnotesize From \textbf{Supervised Machine Learning: A Review of Classification Techniques}, S. B. Kotsiantis, \textit{Informatica}, 31:249--268, 2007.}
\end{frame}

\begin{frame}{Relating your needs and ML}
\small
\visible<1->{
Back to the example of Predictive Maintenance tasks.
\begin{itemize}
\item Visualizing system state\\
\hspace{1cm} $\rightarrow$ Dimensionnality reduction (Unsupervised learning)
\item Detecting anomalies\\
\hspace{1cm} $\rightarrow$ Density estimation (Unsupervised learning)
\item Predicting RUL or TTF\\
\hspace{1cm} $\rightarrow$ Regression (Supervised learning)
\item Predicting failure in $N$ cycles\\
\hspace{1cm} $\rightarrow$ Classification (Supervised learning)
\end{itemize}
}
\visible<2->{
%~\\
\underline{Thinking like a Maintenance Engineer:}\\
How can I monitor my system to manage my maintenance operations?\\
\underline{Thinking like a Data Scientist:}\\
Is this a supervised or an unsupervised problem? What available data?\\
%~\\
\begin{block}{}
Relate this example to your own field.\\
Now you can start discussing with data scientists to design together the most appropriate method for your data and your problem.
\end{block}
}
\end{frame}

\begin{frame}{A word on scikit-learn}
Scikit-learn = Machine Learning in Python
\begin{itemize}
\item Simple and efficient tools for data mining and data analysis
\item Accessible to everybody, and reusable in various contexts
\item Built on NumPy, SciPy, and matplotlib
\item Open source, commercially usable - BSD license
\item Well documented, with lots of examples
\end{itemize}
\url{http://scikit-learn.org}\\
~\\
Let's take a look at the \href{http://scikit-learn.org/stable/user_guide.html}{documentation's table of contents} to grasp a few more keywords.
\end{frame}

\begin{frame}{Course goals}
\small
By the end of the class, you should be able to:
\begin{itemize}
\item implement a generic workflow of data analysis for your application field;
\item know the main bottlenecks and challenges of data-driven approaches;
\item link some field problems to their formal Machine Learning counterparts;
\item know the main categories of Machine Learning algorithms and which formal problem they solve;
\item know the name and principles of some key methods in Machine Learning:
\begin{itemize}
\item SVM,
\item kernel methods,
\item Naive Bayes Classification,
\item Gaussian Processes,
\item Artificial Neural Networks,
\item Deep Learning,
\item Random Forests;
\end{itemize} 
\item know the existence of scikit-learn and its API.
\end{itemize}
\end{frame}

\begin{frame}{What you should expect in the remainder of this class}
\begin{itemize}
\item As many intuitive notions as possible,
\item \ldots but also quite a bit of (hopefully painless) math,\\
\item \ldots and a fair amount of hands-on manipulations and demos.
\end{itemize}
\end{frame}

\begin{frame}{Schedule}
\begin{enumerate}
\item General introduction and motivation \Checkmark \\
{\small \it How does ML fit within your business process.\\
Why you should take time to understand what's under the hood in ML.}
\item The importance of data pre-processing\\
{\small \it A practical illustration.}
\item A geometrical approach to ML\\
{\small \it Support Vector Machines and a bit of kernel theory.}
\item A probabilistic approach to ML\\
{\small \it Naive Bayes Classification and Gaussian Processes.}
\item A connectionist perspective \\
{\small \it From mimicking the human brain to Artificial Neural Networks.\\
Deep Learning.}
\item Ensembles of explainable models\\
{\small \it Decision Trees, Bagging and Random Forests.}
\item Practical session, discussion and conclusion
\end{enumerate}
\end{frame}

\begin{frame}{The importance of data pre-processing}
Images, text, video, sound, measurement time series, continuous or discrete variables, missing data...
\begin{itemize}
\item[$\rightarrow$] filtering out noise and irrelevant data.\\%
{\small \it scaling, filtering, reducing...}
\item[$\rightarrow$] data- and application-specific procedures.\\%
{\small \it domain knowledge leverages non-representative datasets.}
\item[$\rightarrow$] source of potential harm.\\%
{\small \it keep goals in mind, to make informed tradeoffs that might induce bias.}
\end{itemize}
$\Rightarrow$ Crucial elements for a good start.
\begin{center}
Never neglect the pre-processing.
\end{center}
\end{frame}

\begin{frame}{Schedule}
\begin{enumerate}
\item General introduction and motivation \Checkmark \\
{\small \it How does ML fit within your business process.\\
Why you should take time to understand what's under the hood in ML.}
\item The importance of data pre-processing \Checkmark \\
{\small \it A practical illustration.}
\item A geometrical approach to ML\\
{\small \it Support Vector Machines and a bit of kernel theory.}
\item A probabilistic approach to ML\\
{\small \it Naive Bayes Classification and Gaussian Processes.}
\item A connectionist perspective \\
{\small \it From mimicking the human brain to Artificial Neural Networks.\\
Deep Learning.}
\item Ensembles of explainable models\\
{\small \it Decision Trees, Bagging and Random Forests.}
\item Practical session, discussion and conclusion
\end{enumerate}
\end{frame}

\begin{frame}{A geometrical approach to ML}
\begin{block}{}
\begin{enumerate}
\item Draw a line that sits as far as possible from the data points $\rightarrow$ Support Vector Machines
\item Send all data points in a higher dimension space where they are linearly separable $\rightarrow$ kernel trick
\end{enumerate}
$\Rightarrow$ SVM + kernel trick = Find the optimal separating hyperplane in this higher dimension space, without ever computing the mapping.
\end{block}

\begin{itemize}
\item SVM try to separate data by maximizing a geometrical margin
\item They are computed offline
\item They offer a sparse, robust to class imbalance, and easy to evaluate predictor
\item Kernels are a way of enriching (lifting) the data representation so that it becomes linearly separable
\item SVMs + kernels offer a versatile method for classification, regression and density estimation
\item \href{http://scikit-learn.org/stable/modules/svm.html}{Link to documentation in scikit-learn}
\end{itemize}

\end{frame}

\begin{frame}{Schedule}
\begin{enumerate}
\item General introduction and motivation \Checkmark \\
{\small \it How does ML fit within your business process.\\
Why you should take time to understand what's under the hood in ML.}
\item The importance of data pre-processing \Checkmark \\
{\small \it A practical illustration.}
\item A geometrical approach to ML \Checkmark \\
{\small \it Support Vector Machines and a bit of kernel theory.}
\item A probabilistic approach to ML\\
{\small \it Naive Bayes Classification and Gaussian Processes.}
\item A connectionist perspective \\
{\small \it From mimicking the human brain to Artificial Neural Networks.\\
Deep Learning.}
\item Ensembles of explainable models\\
{\small \it Decision Trees, Bagging and Random Forests.}
\item Practical session, discussion and conclusion
\end{enumerate}
\end{frame}

\begin{frame}{A probabilistic approach to ML}
\begin{block}{}
Bayesian approach: find $y$ that maximizes $\mathbb{P}(Y=y|\textrm{data}, X=x)$
\end{block}
This problem of Bayesian inference is hard to solve without additional hypothesis.
\end{frame}

\begin{frame}{A probabilistic approach to ML}
\begin{block}{Naive Bayes classifiers}
\begin{itemize}
\item Make a naive, counter-intuitive hypothesis of conditional independence of the feature variables;
\item Compute each class' probability for a new example using this hypothesis and picks the most probable one;
\item Are a simple, scalable, online method;
\item Despite their simplicity, perform surprisingly well and are competitive in many applications.
\end{itemize}
\end{block}

\end{frame}

\begin{frame}{A probabilistic approach to ML}
\only<1>{
\begin{block}{Gaussian Processes}
\begin{itemize}
\item Compute the most probable function that passes through the data points, given a priori information about how related two data points are (through a covariance kernel);
\item Also provide a measure of prediction uncertainty in each point;
\item Are computed offline and require an $N\times N$ matrix inversion for $N$ data points in the training set (computationnally costly);
\item Careful engineering of covariance kernels can help incorporate priori knowledge into Gaussian Processes;
\item Are suitable both for regression and classification.
\end{itemize}
\end{block}
}
\only<2>{Note that Gaussian Processes are widely used in preliminary design phases, especially as surrogate models that replace physics computations.

\begin{center}
\includegraphics[width=5cm]{img/gaussianprocesses.png}
\end{center}}
\end{frame}

\begin{frame}{Schedule}
\begin{enumerate}
\item General introduction and motivation \Checkmark \\
{\small \it How does ML fit within your business process.\\
Why you should take time to understand what's under the hood in ML.}
\item The importance of data pre-processing \Checkmark \\
{\small \it A practical illustration.}
\item A geometrical approach to ML \Checkmark \\
{\small \it Support Vector Machines and a bit of kernel theory.}
\item A probabilistic approach to ML \Checkmark \\
{\small \it Naive Bayes Classification and Gaussian Processes.}
\item A connectionist perspective \\
{\small \it From mimicking the human brain to Artificial Neural Networks.\\
Deep Learning.}
\item Ensembles of explainable models\\
{\small \it Decision Trees, Bagging and Random Forests.}
\item Practical session, discussion and conclusion
\end{enumerate}
\end{frame}

\begin{frame}{Artificial Neural Networks}
Keywords:
\begin{itemize}
\item Computation graph $f_\theta(x)$
\item Forward pass and gradient backpropagation
\item Online training
\item Minibatches
\item The vanishing gradient problem
\item Keras, Tensorflow, Caffe, Pytorch, Theano
\item Avoiding overfitting: dropout, regularization, data augmentation
\item Convolutional neural networks
\end{itemize}
\end{frame}

\begin{frame}{Artificial Neural Networks}
\begin{itemize}
\item Versatile, online training
\item State-of-the-art performance on many benchmarks
\item But fragile and hard to tune
\item Lots of "recipes" still today
\item Hard to guarantee convergence or performance
\item CNNs = method of choice for structured data (images, sound, time series\ldots)
\end{itemize}
\begin{center}
\includegraphics[width=7cm]{img/IllustrationNeuralNet.pdf}
\end{center}
\end{frame}

\begin{frame}{Schedule}
\begin{enumerate}
\item General introduction and motivation \Checkmark \\
{\small \it How does ML fit within your business process.\\
Why you should take time to understand what's under the hood in ML.}
\item The importance of data pre-processing \Checkmark \\
{\small \it A practical illustration.}
\item A geometrical approach to ML \Checkmark \\
{\small \it Support Vector Machines and a bit of kernel theory.}
\item A probabilistic approach to ML \Checkmark \\
{\small \it Naive Bayes Classification and Gaussian Processes.}
\item A connectionist perspective \Checkmark \\
{\small \it From mimicking the human brain to Artificial Neural Networks.\\
Deep Learning.}
\item Ensembles of explainable models\\
{\small \it Decision Trees, Bagging and Random Forests.}
\item Practical session, discussion and conclusion
\end{enumerate}
\end{frame}

\begin{frame}{Decision trees}
\begin{itemize}
\item Easy to interpret and to explain
\item Poor representative power
\item Greedy growth procedure $\Rightarrow$ suboptimal resulting tree
\item Offline training
\item Very sensitive to noise in the input data
\end{itemize}
\end{frame}

\begin{frame}{Random Forests}
\begin{itemize}
\item RF = decision trees + random feature selection + Bagging
\item Robust, scalable, out-of-the-box classifier
\end{itemize}
$\Rightarrow$ excellent multi-purpose benchmarking algorithm!
\end{frame}

\begin{frame}{Schedule}
\begin{enumerate}
\item General introduction and motivation \Checkmark \\
{\small \it How does ML fit within your business process.\\
Why you should take time to understand what's under the hood in ML.}
\item The importance of data pre-processing \Checkmark \\
{\small \it A practical illustration.}
\item A geometrical approach to ML \Checkmark \\
{\small \it Support Vector Machines and a bit of kernel theory.}
\item A probabilistic approach to ML \Checkmark \\
{\small \it Naive Bayes Classification and Gaussian Processes.}
\item A connectionist perspective \Checkmark \\
{\small \it From mimicking the human brain to Artificial Neural Networks.\\
Deep Learning.}
\item Ensembles of explainable models \Checkmark \\
{\small \it Decision Trees, Bagging and Random Forests.}
\item Practical session, discussion and conclusion
\end{enumerate}
\end{frame}

\begin{frame}{Extra things}
\begin{itemize}
\item Installing anaconda, jupyter, etc.
\item ML research: arXiv, JMLR, MLJ, IEEE PAMI, NeurIPS, ICML, ICLR...
\item datasets: Kaggle, UCI, ImageNet, CIFAR...
\item Other algorithms? (scikit-learn documentation or other notebooks)
\item Dataviz: upstream methods (PCA...) and storytelling (Tableau...)
\item What should I look for in a data scientist's CV?
\item The price to pay for AI: individual responsibility, collective literacy, guiding principles.
\end{itemize}
\end{frame}

\begin{frame}{Schedule}
\begin{enumerate}
\item General introduction and motivation \Checkmark \\
{\small \it How does ML fit within your business process.\\
Why you should take time to understand what's under the hood in ML.}
\item The importance of data pre-processing \Checkmark \\
{\small \it A practical illustration.}
\item A geometrical approach to ML \Checkmark \\
{\small \it Support Vector Machines and a bit of kernel theory.}
\item A probabilistic approach to ML \Checkmark \\
{\small \it Naive Bayes Classification and Gaussian Processes.}
\item A connectionist perspective \Checkmark \\
{\small \it From mimicking the human brain to Artificial Neural Networks.\\
Deep Learning.}
\item Ensembles of explainable models \Checkmark \\
{\small \it Decision Trees, Bagging and Random Forests.}
\item Practical session, discussion and conclusion \Checkmark
\end{enumerate}
\end{frame}

\begin{frame}{Course goals}
\small
By the end of the class, you should be able to:
\begin{itemize}
\item implement a generic workflow of data analysis for your application field;
\item know the main bottlenecks and challenges of data-driven approaches;
\item link some field problems to their formal Machine Learning counterparts;
\item know the main categories of Machine Learning algorithms and which formal problem they solve;
\item know the name and principles of some key methods in Machine Learning:
\begin{itemize}
\item SVM,
\item kernel methods,
\item Naive Bayes Classification,
\item Gaussian Processes,
\item Artificial Neural Networks,
\item Deep Learning,
\item Random Forests;
\end{itemize} 
\item know the existence of scikit-learn and its API.
\end{itemize}
\end{frame}

\end{document}
